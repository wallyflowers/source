\documentclass{article}
\usepackage[margin=1in]{geometry}
\usepackage{graphicx}
\usepackage{hyperref}

\title{Protocol for a Universal Language of Signal Evaluation}
\author{\textit{Life is from everyone, for everyone.}}
\date{\today}

\begin{document}

\maketitle

\begin{abstract}
    PULSE is a distributed network designed to help its users define their core values and to connect users with similar values. It achieves this with a novel protocol for expressing the quality of data within the network, enabling users to have total control over their system of evaluating quality using a shared language, a framework for creating and interpreting a diverse ecosystem of signal types, and interfaces for both language models and human beings. PULSE notably diverges from other network protocols by deliberately not imposing a universal definition of quality upon the network but instead suggesting a distributed language for communicating relative definitions of quality. By allowing nodes to self-organize with other nodes that have resonant ideas of quality data, nodes can form relationships of trust built on the perceived quality of their contributions to the network. Applications built with PULSE have the potential to foster naturally evolving creative communities with shared values, such as self-governing families, companies, or even countries. PULSE explicitly inherits the quality of its design through its inspiration from the enduring products of nature. Because human values are also a product of nature, their convergences and those of other natural phenomena can be viewed as a source of 'divine inspiration.' This defines a paradigm shift in how creative work can be done in technological fields, shining light on the utility of poetic and thematic approaches to the design of computational systems.
\end{abstract}

\section{Introduction}
The invention of data and computing has created two new resources of preeminent value. In the ability to record the nuance of our human expressions with data and to allocate resources to perform the computations we value, we have grown closer to each other and the information created by our environment and ancestry. Still, the ever-increasing and dominant force of globalization has made obvious that a constant stream of ideas is necessary to resolve the tension inherent in our diversity. The internet has created an infrastructure to share data freely across our planet, enabling our expressions to reach each other. The cryptocurrency movement, although still immature, has discovered that it is possible to encode cooperation in software instead of within the mind of a ruling party. The recent progress in large language model artificial intelligence has developed the means to draw on a large portion of human expression to refine new ideas and enable them to be translated into effective language with less effort. What is needed is a unity-seeking decentralized and distributed network with a novel consensus method informed by human values and a central and complementary role for language models to coexist symbiotically with humans. Such a network where diversity is encouraged and values can diverge cooperatively will enable creativity on a global scale.


\section{Layer 1: \textit{The Inner Core}}
- Definition of a 'Signal' and its components (Expression and Quality)
- Explanation of the basic unit of information in the network

\section{Layer 2: \textit{The Outer Core}}
- Introduction to LOVE (Language for Open Value Exchange)
- Explanation of core values and their role in the network
- Key features of LOVE (Universal, Accessible, Simple, and Inclusive)

\section{Layer 3: \textit{The Lower Mantle}}
- Data structures for universal signal processing
- Root network, Trunk, and Form Tree
- How these components work together to process and manage signals

\section{Layer 4: \textit{The Upper Mantle}}
- Common Signal types for networking, RSA, hashing, timestamps, etc.
- Explanation of their purpose and implementation

\section{Layer 5: \textit{The Lithosphere}}
- LanguageModel interfaces
- How these interfaces interact with the rest of the network

\section{Layer 6: \textit{The Crust}}
- Human interfaces
- User interaction and experience with the PULSE network

\section{Use Cases and Applications}
- Potential use cases for the PULSE network
- Examples of how the network can be applied in various domains

\section{Future Work and Roadmap}
- Plans for further development and expansion of the network
- Potential challenges and solutions

\section{Conclusion}
- Summary of the key points and benefits of the PULSE network
- Call to action for participation and contribution

\bibliographystyle{plain}
\bibliography{references}

\end{document}